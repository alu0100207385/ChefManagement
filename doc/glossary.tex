%%%%%%%%%%%%%%%%%%%%%%%%%%%%%%%%%%%%%%%%%%%%%%%%%%%%%%%%%%%%%%%%%%%%%%%%%%%%%
% Chapter: Glosario
%%%%%%%%%%%%%%%%%%%%%%%%%%%%%%%%%%%%%%%%%%%%%%%%%%%%%%%%%%%%%%%%%%%%%%%%%%%%%%%
\setlength{\parindent}{0cm}
{
\textbf{CMS:} [\ref{tabla:clouds}] Es un sistema de gestión de contenidos. Se utiliza para la creación y administración de contenidos, principalmente en páginas web, por parte de los administradores, editores, participantes y demás usuarios. Algunos ejemplos conocidos son \emph{Joomla y Wordpress}.\\

\textbf{CRUD:} [\ref{cap.2.7}] Acrónimo de Crear, Obtener, Actualizar y Borrar. \\

\textbf{Escandallo:} [\ref{cap.1.1}] Cálculo del precio de un producto en relación con sus componentes. En el caso de la restauración, cálculo de producción de una receta en función de sus ingredientes. \\

\textbf{GAE:} [\ref{cap.1.2}] Google App Engine. \\

\textbf{MVC:} [\ref{cap.4.2}] Modelo-vista-controlador. \\

\textbf{Nube (cloud):} [\ref{cap.1.2}] Es el entorno de producción. Se trata de un servidor que ofrece unos determinados servicios o software. \\

\textbf{PaaS:} [\ref{cap.3.1}] Plataforma como servicio.  \\

\textbf{SaaS:} [\ref{cap.3.1}] Software como servicio. \\

\textbf{Scrum:} [\ref{cap.4.1}]  Es un tipo de metodología ágil. \\

\textbf{TDD:} [\ref{cap.1.1}] Desarrollo dirigido por pruebas. \\
}
