%%%%%%%%%%%%%%%%%%%%%%%%%%%%%%%%%%%%%%%%%%%%%%%%%%%%%%%%%%%%%%%%%%%%%%%%%%%%%
% Chapter: Glosario
%%%%%%%%%%%%%%%%%%%%%%%%%%%%%%%%%%%%%%%%%%%%%%%%%%%%%%%%%%%%%%%%%%%%%%%%%%%%%%%

\textbf{Escandallo:}~\ref{sec:cap.1.1} Cálculo del precio de un producto en relación con sus componentes. En el caso de la restauración, cálculo de producción de una receta en función de sus ingredientes. \\

\textbf{CMS:}~\ref{tab:clouds} Es un sistema de gestión de contenidos. Se utiliza para la creación y administración de contenidos, principalmente en páginas web, por parte de los administradores, editores, participantes y demás usuarios. Algunos ejemplos conocidos son \emph{Joomla y Wordpress}.\\

\textbf{CRUD:}~\ref{sec:cap.2.7} Acrónimo de Crear, Obtener, Actualizar y Borrar. \\

\textbf{GAE:}~\ref{sec:cap.1.2} Google App Engine. \\

\textbf{MVC:}~\ref{sec:cap.4.2} Modelo-vista-controlador. \\

\textbf{Nube (cloud):}~\ref{sec:cap.1.2} Es el entorno de producción. Se trata de un servidor que ofrece unos determinados servicios o software. \\

\textbf{SaaS:}~\ref{sec:cap.3.1} Software como servicio. \\

\textbf{TDD:}~\ref{sec:cap.1.1} Desarrollo dirigido por pruebas. \\

\textbf{PaaS:}~\ref{sec:cap.3.1} Platform as a service.  \\

\textbf{Scrum:}~\ref{sec:cap.4.1}  Es un tipo de metodología ágil. \\
