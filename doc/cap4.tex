%%%%%%%%%%%%%%%%%%%%%%%%%%%%%%%%%%%%%%%%%%%%%%%%%%%%%%%%%%%%%%%%%%%%%%%%%%%%%%%
% Chapter 4: Metodología de desarrollo
%%%%%%%%%%%%%%%%%%%%%%%%%%%%%%%%%%%%%%%%%%%%%%%%%%%%%%%%%%%%%%%%%%%%%%%%%%%%%%%

%++++++++++++++++++++++++++++++++++++++++++++++++++++++++++++++++++++++++++++++
En el capítulo~\ref{chapter:intro} se exlpicó la metodología ágil. En este capítulo repasaremos esta metodología usando como ejemplo el desarrollo de \textbf{Chefmanagement} y también el modelo-vista-controlador y su estructura.

\vspace*{0.2in}
\section{Metodología ágil}\label{cap.4.1}

Se había comentado anteriormente el uso de la metodología ágil, esto es debido a la naturaleza del proyecto: se trata de crear una aplicación en un período determinado y cuyas características no están claras, lo cual se debe a que los requisistos no están definidos, no existe un cliente con una necesidad, en este caso se trata de experimentar. Dentro de este tipo de metodologías, se ha usado \emph{Scrum} para obtener resultados rápidamente y donde los requisistos pueden cambiar en cada iteración. La necesidad de innovar y la flexibilidad responden bien con esta metodología. \\

Hay que tener en cuenta, que debido a las circunstancias de este proyecto (no hay un equipo de desarrollo ni un cliente propiamente dicho) se ha adaptado \emph{Scrum} a las necesidades, sin embargo, esta metodología se ha realizado de forma estricta en reuniones con el tutor y en la elaboración de la aplicación. A continucación se explica de forma general una iteración durante el desarrollo de esta aplicación:
\begin{itemize}
	\item \textbf{Planificación:} Cómo me iba a organizar el trabajo durante esa semana. Planificar tiempo a cada una de las siguientes etapas. Corresponde a las reuniones semanales con el tutor y programar los nuevos requisitos en función de los resultados obtenidos.
	\item \textbf{Análisis de requisitos:} Cada vez que se proponía una idea ésta era analizada con el fin de comprobar su viabilidad en la aplicación. Se trata del estudio de campo e investigaciones específicas según la tarea planificada.
	\item \textbf{Diseño:} Cada vez que se desee agregar una funcionalidad al programa se debe estudiar el cambio en el diseño de la misma, de forma que optimice su usabilidad. El diseño también hace referencia a la forma en la que se va a crear los distintos modelos de la base de datos y sus relaciones. Crear la estructura del proyecto de forma que sea escalable.
	\item \textbf{Codificación:} Parte del tiempo en el que se desarrollaba código. Durante el estudio de campo esta etapa se destino a crear programas sencillos de prueba.
	\item \textbf{Revisión:} Comprobar la aplicación hasta el punto actual. Tanto las características nuevas como las anteriores. También se incluyen las pruebas de código.
	\item \textbf{Documentación:} Durante el tiempo de vida del proyecto se ha ido elaborando un diario, en él se guarda la información de los resultados obtenidos semanalmente y expuestos durante las reuniones: lo qué se ha averiguado, lo qué se ha conseguido, los problemas encontrados y las soluciones propuestas.
\end{itemize}

\vspace*{0.2in}
\section{Estructura de la app}\label{cap.4.2}

MVC
CRUD
MVC: Ruby Sinatra
Rails VS Sinatra
