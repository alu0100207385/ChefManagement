\documentclass[a4paper,openright,14pt]{book}
\usepackage[spanish]{babel}
\selectlanguage{spanish}
\usepackage[utf8]{inputenc}
\usepackage{hyperref}
%%%%%%%%%%%%%%%%%%%%%%%%%%%%%%%%%%%%%%%%%%%%%%%%%%%%%%%%%%%%%%%%%%%%%%%%%%%%%
% Chapter 1: Introducción 
%%%%%%%%%%%%%%%%%%%%%%%%%%%%%%%%%%%%%%%%%%%%%%%%%%%%%%%%%%%%%%%%%%%%%%%%%%%%%%%

%---------------------------------------------------------------------------------
\begin{document}
\chapter{Introducción}\label{intro}
\pagenumbering{arabic}

\section{Objetivos}\label{cap.1.1}
El objetivo de este trabajo de fin de grado (en adelante, TFG) es el desarrollo de una aplicación web para la gestión del escandallo, es decir, controlar el precio de producción de una receta. Para ello usaremos Cloud Servers, concretamente Google Cloud Platform, sin embargo, por algunas de las restricciones que mostraba, las cuales se describirán más adelante, se reorientó el TFG hacia otras plataformas como servicio (SaaS). Para su desarrollo usé la metodología de software ágil basada en el desarrollo dirigido por pruebas (TDD). Como framework, Sinatra (ruby) para desarrollar la aplicación, integrar APIs de terceros, y hacer uso de varias nubes de producción a la vez y también herramientas de soporte para el desarrollo de software como son Github y Travis, entre otras.

\vspace*{0.2in}
\section{Antecedentes y estado actual del tema}\label{cap.1.2}
Los antecedentes y estudio de campo no sólo se centran en el tipo de aplicación que vamos a desarrollar, también tendremos en cuenta el entorno de producción, es decir, la nube donde estará disponible la aplicación web. Por lo tanto, tenemos dos puntos iniciales de estudio: 
\begin{itemize}
	\item Sobre la aplicación: 
	\begin{itemize}
		\item ¿Qué ofrece y por qué? 
		\item ¿Qué otras aplicaciones similares existen?
		\item Estudio del sector del mercado.
		\item Soporte y aspectos de la usabilidad.
	\end{itemize}

	\item Y respecto a la nube:
	\begin{itemize}
		\item ¿Cuál es la mejor nube que se adapte a nuestras necesidades?
		\item ¿Qué lenguajes y frameworks soporta?
		\item Período, costos, capacidad, etc.
	\end{itemize}
\end{itemize}

\vspace*{0.2in}
\begin{LARGE}
	\textbf {Aplicación}
\end{LARGE}

\vspace*{0.1in}
La finalidad de esta aplicación nace de la necesidad de optimizar los gastos, tanto en negocios de restauración como en los propios hogares. Hacer la comida controlando gastos es posible. En la actualidad existen en el mercado aplicaciones con esta idea, algunas de las mas destacadas son:
\begin{itemize}
	\item \href{https://recipecostcalculator.net/}{Recipe Cost Calculator}: Quizás sea la aplicación base como referencia, su fácil gestión y múltiples funciones la hace una herramienta útil y potente. Esta principalmente orientada a negocios.
	\item \href{http://www.recipe-costing.com/}{Recipe Costing}: Esta aplicación va más allá de las funcinalidades básicas, presenta extras como cálculos de menús, gestión de inventario, órdenes de compra a proveedores, etc.
	\item \href{http://developer.pearson.com/apis/pearson-kitchen-manager}{Pearson Kitchen Manager}: Se trata de una API, un banco de información que contiene más de 3000 mil recetas etiquetadas y con otra información como sus valores nutricionales.
	\item \href{https://itunes.apple.com/es/app/recipe-costing-calculator/id646877156?mt=8}{Recipe Costing Calculator}: Es más sencilla que las anteriores pero se trata de App disponible en iTunes. Las cosas más sencillas puden ser las más útiles, hay que tener en cuenta que un cocinero/a prefiere una tablet a un ordenador en la cocina.
\end{itemize}
La segmentación del mercado está orientado especialmente a negocios de restauración: comedores, restaurantes, bares, pastelerías, panaderías, etc. Por otro lado, existe otro sector que no se tiene tan en cuenta debido a que no genera tantos beneficios, se trata de los hogares. Las personas también pueden hacer uso de esta herramienta pues sus necesidades son las mismas pero a menor escala. En el próximo capítulo~\ref{chapter:dos}, veremos las diferencias en la aplicación en función de a qué mercado está dirigido.\\

Por último, hay que tener en cuenta el soporte para la aplicación. Aprovechando la infraestructura y poder de Internet, la mejor opción es crear una aplicación web, que aunque en esta primera versión se diseñará de forma adaptativa, la idea es poder utilizarla en el futuro en dispositivos móviles y tablets. Además, se ha tenido en cuenta los aspectos de usabilidad durante su diseño.

\vspace*{0.3in}
\begin{LARGE}
	\textbf{{\huge Nube (cloud)}}
\end{LARGE}

\vspace*{0.1in}
Inicialmente la idea es trabajar en la \href{https://cloud.google.com/appengine/docs}{nube de Google}. Se trata de una plataforma como servicio (PaaS), la cual permite crear y mantener de forma sencilla una aplicación en la infraestructura de Google. Además permite una fácil escalabilidad de transeferencia de datos y almacenamiento gracias a sus módulos.\\

Aprovechando los conocimientos adquiridos durante los últimos cursos en ruby y sus variedad de frameworks (Ruby on Rails, Padrino, Sinatra) lo usaré para crear la aplicación. Tras investigar y ver los servicios que ofrece Google App Engine (GAE) parece viable la puesta en marcha de la aplicación. GAE soporta cuatro lenguajes y con correspondientes frameworks:
\begin{itemize}
	\item Python con webapp2 y Jinja2.
	\item Java con maven.
	\item PHP con Cloud SQL.
	\item Go con el paquete html/plantilla. 
\end{itemize}
Sin embargo, también es posible hacerlo funcionar en ruby con la ayuda de java, juntos forman \href{http://jruby.org/}{Jruby}.
		%\item ¿Cuál es la mejor nube que se adapte a nuestras necesidades?
		%\item ¿Qué lenguajes y frameworks soporta?
		%\item Período, costos, capacidad, etc.

\vspace*{0.2in}
\section{Metodología de trabajo}\label{cap.1.3}


\end{document}



%---------------------------------------------------------------------------------
\begin{comment}

\label{1:sec:1}


Citando una entrada de memtfg.bib \cite{URL:GitHub}

Incluyento una gráfica y referenciandola como Figura \ref{fig:prueba} y tambien un poco de código como Listado \ref{code:prueba}

\begin{figure}[t]
  \begin{center}
    \includegraphics[width=0.4\textwidth]{logotipo-secundario-ULL}
  \end{center}
  \caption{Este es un ejemplo de figura. Mira config de graphicx y  graphicspath}
  \label{fig:prueba}
\end{figure}

\lstinputlisting[float=t,
                 caption={Listado de codigos blabla},
                 label={code:prueba}]
                 {codes/model.rb}
\end{comment}

%---------------------------------------------------------------------------------
\begin{comment}
\section{XXXX}
\label{1:sec:2}

En el Capítulo \ref{chapter:tres}
se describe toda la funcionalidad de la aplicación, al igual que nos enseña como usarla.\\
\end{comment}