%%%%%%%%%%%%%%%%%%%%%%%%%%%%%%%%%%%%%%%%%%%%%%%%%%%%%%%%%%%%%%%%%%%%%%%%%%%%%
% Chapter 1: Introducción 
%%%%%%%%%%%%%%%%%%%%%%%%%%%%%%%%%%%%%%%%%%%%%%%%%%%%%%%%%%%%%%%%%%%%%%%%%%%%%%%
El objetivo de este proyecto es 

%---------------------------------------------------------------------------------
\section{Antecedentes y estado actual del tema}
\label{1:sec:1}


Citando una entrada de memtfg.bib \cite{URL:GitHub}

Incluyento una gráfica y referenciandola como Figura \ref{fig:prueba} y tambien un poco de código como Listado \ref{code:prueba}

\begin{figure}[t]
  \begin{center}
    \includegraphics[width=0.4\textwidth]{logotipo-secundario-ULL}
  \end{center}
  \caption{Este es un ejemplo de figura. Mira config de graphicx y  graphicspath}
  \label{fig:prueba}
\end{figure}

\lstinputlisting[float=t,
                 caption={Listado de codigos blabla},
                 label={code:prueba}]
                 {codes/model.rb}


%---------------------------------------------------------------------------------
\section{XXXX}
\label{1:sec:2}

En el Capítulo \ref{chapter:tres}
se describe toda la funcionalidad de la aplicación, al igual que nos enseña como usarla.\\

