%%%%%%%%%%%%%%%%%%%%%%%%%%%%%%%%%%%%%%%%%%%%%%%%%%%%%%%%%%%%%%%%%%%%%%%%%%%%%
% Chapter 1: Introducción 
%%%%%%%%%%%%%%%%%%%%%%%%%%%%%%%%%%%%%%%%%%%%%%%%%%%%%%%%%%%%%%%%%%%%%%%%%%%%%%%
El objetivo de este trabajo de fin de grado (en adelante, TFG) es el desarrollo de una aplicación web para la gestión del escandallo, es decir, controlar el precio de producción de una receta. Para ello usaremos Cloud Server, concretamente Google Cloud Platform, sin embargo, por algunas de las restricciones que mostraba, las cuales se describirán más adelante, se reorientó el TFG hacia otras plataformas como servicio (SaaS). Para su desarrollo usé la metodología de software ágil basada en el desarrollo dirigido por pruebas. Como framework, Sinatra (ruby) para desarrollar la aplicación, integrar APIs de terceros, y hacer uso de varias nubes de producción a la vez y también herramientas de soporte para el desarrollo de software como son github y travis, entre otras.

Palabras clave: TFG, escandallo, Cloud, SaaS, API.
%---------------------------------------------------------------------------------
\section{Antecedentes y estado actual del tema}
\label{1:sec:1}


Citando una entrada de memtfg.bib \cite{URL:GitHub}

Incluyento una gráfica y referenciandola como Figura \ref{fig:prueba} y tambien un poco de código como Listado \ref{code:prueba}

\begin{figure}[t]
  \begin{center}
    \includegraphics[width=0.4\textwidth]{logotipo-secundario-ULL}
  \end{center}
  \caption{Este es un ejemplo de figura. Mira config de graphicx y  graphicspath}
  \label{fig:prueba}
\end{figure}

\lstinputlisting[float=t,
                 caption={Listado de codigos blabla},
                 label={code:prueba}]
                 {codes/model.rb}


%---------------------------------------------------------------------------------
\section{XXXX}
\label{1:sec:2}

En el Capítulo \ref{chapter:tres}
se describe toda la funcionalidad de la aplicación, al igual que nos enseña como usarla.\\
