%%%%%%%%%%%%%%%%%%%%%%%%%%%%%%%%%%%%%%%%%%%%%%%%%%%%%%%%%%%%%%%%%%%%%%%%%%%%%
% Chapter 5: Conclusiones y Trabajos Futuros 
%%%%%%%%%%%%%%%%%%%%%%%%%%%%%%%%%%%%%%%%%%%%%%%%%%%%%%%%%%%%%%%%%%%%%%%%%%%%%%%

%++++++++++++++++++++++++++++++++++++++++++++++++++++++++++++++++++++++++++++++
La idea de crear una aplicación en la nube puede ser más engorroso que el desarollo de aplicaciones locales. Esto significa programar en varios entornos y cada uno con sus propias reglas. El MVC hace esto más sencillo. \\

Gracias a la experiencia adquirida en las prácticas de empresa he dado importancia al estudio de mercado, lo que me ha llevado a entrevistas con distintos restaurantes y proveedores y saber sus necesidades sobre la aplicación. Así mismo, la idea de innovar en este tipo de aplicaciones incluye un estudio de campo de las mismas. \\

También hay que tener en cuenta la investigación de las nubes de producción, las pruebas y analizar cual se adapta mejor a las necesidades de este proyecto. \\

Resumiendo este proyecto recoge protocolos y meotodlogías como: Scrum (metodología ágil), uso de frameworks basado en MVC, TDD (locales y en la servidores remotos), uso de repositorios y programación en la nube. \\

Se ha introducido una nueva PaaS (OpenShift) y se ha desarrollado una nueva aplicación con perspectivas de seguir evolucionando, procurando un diseño escalable e incluso exportable a otras arquitecturas.

Citando una entrada de memtfg.bib \cite{URL:GitHub}


\begin{center}
	\rule{100mm}{0.2mm}\\
\end{center}

A continuación se propone mejoras para incluir en módulos de futuras versiones de la aplicación:

\begin{itemize}
\item Integrar APIs de precios de productos de proveedores (Mercadona, Alcampo, etc). Esto permite un precio real y actualizado de la receta. Aunque no está integrada en la aplicación se propone usar el \href{http://www.opendatacanarias.es/datos/dataset/mercatenerife-precios-de-productos-hortofruticolas-de-tenerife}{listado de productos} (en formato json) de \textbf{Mercatenerife}.

\item Actualizar y mejorar el diseño con temas personalizables.
\item Incluir en el modelo información sobre el plato (kcal, proteínas, hidratos, grasas, etc.).
\item Añadir a la funcionalidad \emph{Calculadora} evaluar los \textbf{costos de menús}.
\item Desarrollo en dispositivos smatphones y tablets, aprovechando la compatibilidad con JRuby se propone desarrollar en \textbf{Ruboto}.

\item La aplicación puede diversificarse atendiendo a su mercado: \textbf{profesional} (negocios de restauración) o enfoque de \textbf{red social}.
	\item Para el mercado de red social se propone:
		\begin{itemize}
			\item \textbf{Sistema de puntuación}: comprar recetas o conseguir a través de las puntuaciones que consigas compartiendo tus recetas.
			\item \textbf{Otras características de red social}: compartir recetas, marcar como favoritos, comentarios, etc.
			\item \textbf{Estadísticas}: platos-puntos, usuarios-recetas, platos-precio, etc.
		\end{itemize}
	\item Para el mercado de negocio:
		\begin{itemize}
			\item \textbf{Buscar} un determinado ingrediente en un determinado momento. Ver el precio actual.
			\item \textbf{Comprar} al proveedor a través de la aplicación.
		\end{itemize}

\end{itemize}
