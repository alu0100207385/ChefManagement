%%%%%%%%%%%%%%%%%%%%%%%%%%%%%%%%%%%%%%%%%%%%%%%%%%%%%%%%%%%%%%%%%%%%%%%%%%%%%
% Chapter 6: Summary and Conlusions
%%%%%%%%%%%%%%%%%%%%%%%%%%%%%%%%%%%%%%%%%%%%%%%%%%%%%%%%%%%%%%%%%%%%%%%%%%%%%%%

%++++++++++++++++++++++++++++++++++++++++++++++++++++++++++++++++++++++++++++++

Over the course of this document we have seen different aspects associated with the development of \textbf{Chefmanagement}, using agile methodology. It has been used the knowledge gained during academic education, this includes research and analysis and actual practice in projects and companies. \\

Thanks to the experience gained in enterprise practices I have given importance to market research, this has led me to interviews with various restaurants and suppliers to know their needs on the application. Also, the idea of innovation in this type of application includes a field study of them. \\

We must also take into account the investigation of production clouds, tests and analyse which is best suited to this project needs. \\

In short, this project includes protocols and methodologies such as Scrum (Agile methodology), using frameworks based on MVC, TDD (local and on remote servers) use of repositories and cloud programming. \\

It has been introduced a new PaaS (OpenShift) and a new application has been developed in order to continue evolving, trying to design a scalable and even exportable to other architectures.