%%%%%%%%%%%%%%%%%%%%%%%%%%%%%%%%%%%%%%%%%%%%%%%%%%%%%%%%%%%%%%%%%%%%%%%%%%%%%
% Chapter 6: Summary, conlusions and future works
%%%%%%%%%%%%%%%%%%%%%%%%%%%%%%%%%%%%%%%%%%%%%%%%%%%%%%%%%%%%%%%%%%%%%%%%%%%%%%%

%++++++++++++++++++++++++++++++++++++++++++++++++++++++++++++++++++++++++++++++

Over the course of this document we have seen different aspects associated with the development of \textbf{Chefmanagement}, using agile methodology. It has been used the knowledge gained during academic education, this includes research and analysis and actual practice in projects and companies. \\

Thanks to the experience gained in enterprise practices I have given importance to market research, this has led me to interviews with various restaurants and suppliers to know their needs on the application. Also, the idea of innovation in this type of application includes a field study of them. \\

We must also take into account the investigation of production clouds, tests and analyse which is best suited to this project needs. \\

In short, this project includes protocols and methodologies such as Scrum (Agile methodology), using frameworks based on MVC, TDD (local and on remote servers) use of repositories and cloud programming. \\

It has been introduced a new PaaS (OpenShift) and a new application has been developed in order to continue evolving, trying to design a scalable and even exportable to other architectures.

\newpage

Next it will suggest upgrades to include in future version of this application:

\begin{itemize}
\item Integrate APIs to get prices of products from providers (Mercadona, Alcampo, etc). This allows a real price and updated of ingredients to use in recipes. It suggested to use (\cite{URL:Mercatenerife_Precio_Productos}) (json format) de \textbf{Mercatenerife's} product list.

\item Update and improve application design adding customizable themes.
\item Modify Recipe model and add nutrition information (kcal, proteins, carbohydrates, fats, etc.).
\item Add to \emph{Calculator's} functionality that can calculate \emph{menu prices}.
\item Development application to use in smatphones devices and tablets, take advantage of compatibility with JRuby (It is recommended \textbf{Ruboto}).
\item The application can be diversified according to current market: \textbf{profesional} (catering business) or \textbf{social network}.
	\item To social network market it propose:
		\begin{itemize}
			\item \textbf{Scoring System}: you can buy recipes or get them exchanging your points that you can get if you share your recipes.
			\item \textbf{Other features of the social network}: you can share recipes, mark like favourite, comments, etc.
			\item \textbf{Statistics}: dish-points, users-recipes, dish-price, etc.
		\end{itemize}
	\item To profesional market:
		\begin{itemize}
			\item \textbf{Search:} you can look for an ingredient on a certain date. You can see its current price.
			\item \textbf{Buy:} you can get ingredients from provider across the application.
		\end{itemize}

\end{itemize}
