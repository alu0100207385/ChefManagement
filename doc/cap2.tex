\documentclass[a4paper,14pt]{book}
\usepackage[spanish]{babel}
\selectlanguage{spanish}
\usepackage[utf8]{inputenc} 
%%%%%%%%%%%%%%%%%%%%%%%%%%%%%%%%%%%%%%%%%%%%%%%%%%%%%%%%%%%%%%%%%%%%%%%%%%%%%%%
% Chapter 2: Aplicación ChefManagement
%%%%%%%%%%%%%%%%%%%%%%%%%%%%%%%%%%%%%%%%%%%%%%%%%%%%%%%%%%%%%%%%%%%%%%%%%%%%%%%
\begin{document}
\chapter{Introducción}\label{dos}
\pagenumbering{arabic}

En el capítulo anterior~\ref{chapter:intro} se han introducido tanto los antecedentes como se describió brevemente la aplicación. En este capítulo explicaremos toda las funcionalidades y requisitos.

Partimos de qué tipo de \textbf{aplicación} queremos crear. Se trata de desarrollar un software que ayude a las personas a gestionar el costo de producción de las recetas, dependiendo del precio de sus ingredientes y el número de raciones. Las características iniciales son:
\begin{itemize}
	\item Crear, ver, editar y eliminar recetas.
	\item Listar todas las recetas.
	\item Crear backup de las recetas y cargarlo si es necesario.
	\item Calcular el precio de una determinada receta para un determinado número de comensales.
\end{itemize}

Además para esta primera versión, debe ser capaz de realizar una serie de tareas mínimas, que cumplan con las necesidades del usuario y faciliten su interacción con la aplicación. Éstos son sus requisitos iniciales:
\begin{itemize}
	\item Dos entornos de producción: la aplicación puede ser usada desde dos nubes diferentes.
	\item Tener en cuenta aspectos de usablidad: adaptación, entendimiento y facilidad de uso.
	\item Importar y exportar archivos en formato json: almacenar copias de las recetas en el equipo cliente y restaurarlas si es necesario.
	\item Accesso mediante registro o haciendo uso de APIs de distintas redes sociales como Google+ o Facebook. Hay que facilitar el acceso a los usuarios.
\end{itemize}

Tanto las características comos los requisitos de la aplicación serán aplicadas a la versión 1.0, pudiendo ser mejorados o añadidos más en futuras versiones~\ref{chapter:cinco}.

\end{document}
